% ------------------------------------------------------------------------
% ------------------------------------------------------------------------
% UFGRC: Modelo de Trabalho Acadêmico em conformidade com 
% ABNT NBR 14724:2011: Informação e documentação - Trabalhos acadêmicos -
% Apresentação
% ------------------------------------------------------------------------
% ------------------------------------------------------------------------

% Opções: 
%   Tipo do trabalho     = tcc1/tcc2
%   Situação do trabalho = pre-defesa/pos-defesa
% -- opções do pacote babel --
% Idioma padrão = brazil
	%english,			% idioma adicional para hifenização
	%brazil				% o último idioma é o PRINCIPAL do documento
\documentclass[tcc2, pos-defesa, english, brazil]{packages/ufgrc}

% ---------------------------------------------------------------------------
% Pacotes Opcionais
% ---------------------------------------------------------------------------
\usepackage{rotating}           % Usado para rotacionar o texto
\usepackage[all,knot,arc,import,poly]{xy}   % Pacote para desenhos gráficos
% Este pacote pode conflitar com outros pacotes gráficos como o ``pictex''
% Então é necessário usar apenas um dos pacotes conflitantes
\newcommand{\VerbL}{0.52\textwidth}
\newcommand{\LatL}{0.42\textwidth}
% ---------------------------------------------------------------------------


% ---
% Informações de dados para CAPA e FOLHA DE ROSTO
% ---
% Tanto na capa quanto nas folhas de rosto apenas a primeira letra da primeira palavra (ou nomes próprios) devem estar em letra maiúscula, todas as demais devem ser em letra minúscula.
\titulo{Modelo de monografia em LaTeX do DC/UFG -- Regional Catalão}
\autor[Antonelli, H. L.]{Humberto Lidio Antonelli}
\genero{M} % Gênero do autor (M = Masculino / F = Feminino)
\orientador[Orientadora]{Prof.$^a$ Dr.$^a$}{Núbia Rosa da Silva}
%\coorientador{Prof. Dr.}{Fulano de Tal}
\data{03}{10}{2016} % Data da defesa
% ---

% Membros da banca examinadora
% - O primeiro membro será automaticamente o orientador
% - Caso haja coorientador, este será o segundo membro
% Nome dos demais membros e suas instituições
\membrobanca{Fulano de Tal}{Instituição do Fulano de Tal}
\membrobanca{Ciclano de Tal}{Instituição do Ciclano de Tal}

% ---
% RESUMOS
% ---

% Resumo em PORTUGUÊS
% conter no máximo 500 palavras
% conter no mínimo 1 e no máximo 5 palavras-chave (obrigatoriamente separadas por vírgula)
\textoresumo[brazil]{
    Este trabalho é um breve modelo para a escrita de monografias de trabalhos de conclusão de curso utilizando o ambiente \LaTeX, de acordo com as normas exigidas pelo curso de Ciência da Computação, da Unidade Acadêmica Especial de Biotecnologia, da Universidade Federal de Goiás (UFG). Para a confecção deste modelo foi utilizado a última versão (1.9.6) do pacote de classes \textit{abnTeX2} que segue as normas da Associação Brasileira de Normas Técnicas. A elaboração de uma monografia, dissertação ou tese pode ser feita sobrescrevendo o conteúdo deste modelo. 
    }{Modelo, Monografia, Trabalho de Conclusão de Curso, Latex}


% resumo em INGLÊS
% conter no máximo 500 palavras
% conter no mínimo 1 e no máximo 5 palavras-chave (obrigatoriamente separadas por vírgula)
\textoresumo[english]{
    This paper is a brief model for writing monographs of Final Papers using \LaTeX environment, in accordance with the standards required by the Course of Computer Science, Unidade Acadêmica Especial de Biotecnologia, Federal of University of Goiás (UFG). For making this model, the latest version (1.9.6) \textit{abnTeX2} classes package was used. This package follow the rules of the Brazilian Association of Technical Standards. A drafting a monograph, dissertation or thesis can be done by overwriting the contents of this model.
    }{Template, Monograph, Term Paper, Latex}
    
% ---
% Configurações de aparência do PDF final
% ---
\hypersetup{
	colorlinks=true     % false: boxed links; true: colored links
}
% --- 

% ----------------------------------------------------------
% ELEMENTOS PRÉ-TEXTUAIS
% ----------------------------------------------------------

% Inserir a ficha catalográfica
%\incluifichacatalografica*{tex/pre-textual/fichaCatalografica.pdf}
\incluifichacatalografica

% DEDICATÓRIA / AGRADECIMENTO / EPÍGRAFE
\textodedicatoria*{tex/pre-textual/dedicatoria}
\textoagradecimentos*{tex/pre-textual/agradecimentos}
\textoepigrafe*{tex/pre-textual/epigrafe}

% Inclui a lista de figuras
\incluilistadefiguras

% Inclui a lista de tabelas
\incluilistadetabelas

% Inclui a lista de quadros
\incluilistadequadros

% Inclui a lista de algoritmos
\incluilistadealgoritmos

% Inclui a lista de códigos
\incluilistadecodigos

% Inclui a lista de siglas e abreviaturas
\incluilistadesiglas

% Inclui a lista de símbolos
\incluilistadesimbolos

% ----
% Início do documento
% ----
\begin{document}

% ----------------------------------------------------------
% ELEMENTOS TEXTUAIS
% ----------------------------------------------------------
\textual

\chapter{Introdução}
\label{chapter:introducao}
% Comando simples para exibir comandos Latex no texto
\newcommand{\comando}[1]{\textbf{$\backslash$#1}}

Este documento explica brevemente como trabalhar com a classe \LaTeX~\textit{ufgrc} para confeccionar trabalhos acadêmicos seguindo as normas da \sigla{ABNT}{Associação Brasileira de Normas Técnicas}\cite{NBR14724:2011}. O presente manual visa atender as exigências da \sigla{UFG}{Universidade Federal de Goiás}.


A classe \textit{ufgrc} foi construída com base na última versão da classe \textit{abntex2} e do pacote \textit{abntex2cite}. Portanto, este documento exemplifica a elaboração de trabalho
acadêmico (tese, dissertação e outros do gênero) produzido conforme a ABNT NBR
14724:2011 \textit{Informação e documentação - Trabalhos acadêmicos - Apresentação}.

Assim, é altamente recomendável que seja consultada a documentação do \textit{abntex2}\footnote{http://abntex.net.br}. A classe \textit{abntex2} foi desenvolvida para facilitar a escrita de documentos seguindo as normas da ABNT no ambiente \LaTeX\;\cite{frasson:2005:classe_abnt}.

Todo o trabalho de pesquisa e ajustes da presente classe \LaTeX~\emph{ufgrc} foram feitos pelo ex-aluno de Ciências da Computação da UFG -- Regional Catalão, Humberto Lidio Antonelli, que colaborou para o desenvolvimento do modelo utilizado anteriormente pelo curso.

O requisito básico para utilização da classe \textit{ufgrc} é criar um documento desta classe com o comando
\comando{documentclass[@parameters]\{ufgrc\}} e ter, no diretório de trabalho, o arquivo \emph{ufgrc.cls} presente. Entretanto, recomenda-se fortemente manter a estrutura de diretório inicial fornecida por este modelo. Os parâmetros possíveis utilizados pelo \comando{documentclass} são:
\begin{description}
    \item[tcc1 / tcc2] Identifica a etapa de trabalho correspondente ao qual o aluno está, sendo utilizado apenas uma das duas opcões disponíveis. O valor padrão é \textbf{tcc2};
    
    \item[pre-defesa / pos-defesa] Identifica a situação do documento (exceto para qualificação), sedo necessário apenas uma das duas opções. O valor padrão é \textbf{pos-defesa};
    
    \item[french, spanish, english, brazil] Adiciona o idioma para correta hifenização correta no documento. O último idioma declarado é o principal do documento. O valor padrão é \textbf{brazil}.
\end{description}

{\color{red}{Embora o idioma português-brasileiro (\texttt{brazil}) seja incluído automaticamente pela classe \textsf{abntex2}, a opção \texttt{brazil} deve OBRIGATORIARMENTE ser informada para o correto funcionamento do \textit{template} para uso no idioma em inglês, de modo que todos os pacotes reconheçam ambos os idiomas.}}


\chapter{Instalando o abnTeX2}
\label{chapter:instalando-abntex}
\input{tex/instalando-abntex}

\chapter{Orientações gerais}
\label{chapter:orientacoes-gerais}
\input{tex/orientacoes-gerais}

\chapter{Configuração dos elementos pré-textuais}
\label{chapter:config-pre-textual}
\input{tex/config-pre-textual}

\chapter{Corpos flutuantes}
\label{chapter:corpos-flutuantes}
\input{tex/corpos-flutuantes}

\chapter{Listas}
\label{chapter:listas}
\input{tex/listas}

\chapter{Ferramentas úteis}
\label{chapter:ferramentas-uteis}
\input{tex/ferramentas-uteis}

\chapter{Citações e referências}
\label{chapter:citacoes}
\input{tex/citacoes}


% ---
% Finaliza a parte no bookmark do PDF, para que se inicie o bookmark na raiz
% ---
\bookmarksetup{startatroot}% 
% ---

% ----------------------------------------------------------
% ELEMENTOS PÓS-TEXTUAIS
% ----------------------------------------------------------
\postextual

% ----------------------------------------------------------
% Referências bibliográficas
% ----------------------------------------------------------
\bibliography{references}

% ---------------------------------------------------------------------
% GLOSSÁRIO
% ---------------------------------------------------------------------

% Arquivo que contém as definições que vão aparecer no glossário
\input{tex/glossario}
% Comando para incluir todas as definições do arquivo glossario.tex
\glsaddall
% Impressão do glossário
\printglossaries

% ----------------------------------------------------------
% Apêndices
% ----------------------------------------------------------

% ---
% Inicia os apêndices
% ---
\begin{apendicesenv}

    \chapter{Documento básico usando a classe \textit{ufgrc}}
    \label{chapter:documento-basico}
    \input{tex/appendix/documento-basico}
    
    \chapter{Configuração do programa JabRef}
    \label{chapter:configuracao-jabref}
    \input{tex/appendix/configuracao-jabref}

\end{apendicesenv}
% ---


% ----------------------------------------------------------
% Anexos
% ----------------------------------------------------------

% ---
% Inicia os anexos
% ---
\begin{anexosenv}

    \chapter{Páginas interessantes na Internet} 
    \label{chapter:paginas-interessantes}
    \input{tex/annex/paginas-interessantes}

\end{anexosenv}
% ---

\end{document}